\chapter{طراحی سیستم و پیاده‌سازی} % Chapter 3
\label{ch:system-design-implementation}

% ==============================================================================================================
\section{انتخاب فرکانس و پهنای باند مناسب} % Section 3.1
\label{sec:freq-bandwidth-selection}

انتخاب صحیح فرکانس حامل و پهنای باند سیگنال \lr{FMCW} از عوامل کلیدی در طراحی یک سیستم دقیق و قابل اعتماد برای اندازه‌گیری ضربان قلب به روش غیرتماسی است. این انتخاب نه تنها دقت تفکیک فاصله و حساسیت به ارتعاشات سطحی بدن را تحت تأثیر قرار می‌دهد، بلکه در عملکرد سیستم در شرایط واقعی، مانند حضور نویز محیطی و حرکت‌های تنفسی، نیز بسیار تعیین‌کننده است.

\subsection{انتخاب فرکانس} % Subsection 3.1.1
\label{sec:frequency-selection}

فرکانس‌های مورد استفاده در سیستم‌های رادار \lr{FMCW} معمولاً در باندهای میلی‌متری (\lr{mmWave}) قرار دارند، که شامل \lr{24 گیگاهرتز}، \lr{60 گیگاهرتز}، \lr{77 گیگاهرتز}، و حتی تا \lr{140 گیگاهرتز} می‌شود. انتخاب فرکانس بالاتر موجب افزایش دقت تفکیک مکانی و حساسیت به ارتعاشات کوچک مانند ضربان قلب می‌شود؛ اما در مقابل، نفوذ کمتر در بافت‌ها و تأثیرپذیری بیشتر از موانع فیزیکی را به دنبال دارد.

در مقاله "\lr{A Novel Scheme of High-Precision Heart Rate Detection with a mm-Wave FMCW Radar}"، از راداری با فرکانس \lr{77 گیگاهرتز} استفاده شده است. دلیل این انتخاب، تعادل مناسب بین پوشش برد متوسط و حساسیت بالا در تشخیص تغییرات میکرومتری سطح قفسه سینه بوده است. در این پژوهش، با استفاده از یک زنجیره پردازشی شامل فیلتر تطبیقی، انتخاب دامنه با واریانس بالا، و تخمین فرکانسی با \lr{Double-Chirp Z-Transform}، دقت اندازه‌گیری ضربان قلب به کمتر از \lr{1} ضربه در دقیقه رسید.

از سوی دیگر، مقاله "\lr{Short-time Heart Rate Estimation Based on 60-GHz FMCW Radar}" از باند \lr{60 گیگاهرتز} استفاده کرده که به‌طور معمول برای تجهیزات تجاری مقرون‌به‌صرفه‌تر و با ابعاد فیزیکی کوچک‌تر مناسب است. مزیت \lr{60 گیگاهرتز} در این پژوهش، توانایی در اندازه‌گیری سریع در پنجره‌های زمانی \lr{5 ثانیه‌ای} است. استفاده از ترکیب الگوریتم‌های \lr{ICEEMDAN} و \lr{Fast-ICA} برای حذف مؤلفه‌های تنفسی، این امکان را فراهم کرده که ضربان قلب با خطای کمتر از \lr{4 bpm} حتی در حضور نویز تنفسی قابل استخراج باشد.

در سطحی بالاتر، مقاله "\lr{Continuous Radar-based Heart Rate Monitoring using Autocorrelation-based Algorithm in ICU}" به بررسی عملکرد یک سیستم \lr{140 گیگاهرتزی} می‌پردازد. در این پژوهش، رادار در زیر تخت بیماران \lr{ICU} قرار گرفته و توانسته است ضربان قلب را با دقت بالا (\lr{MAE} \lr{$\approx 2.2$ bpm}) در محیط‌های بالینی پرنویز اندازه‌گیری کند. فرکانس بالا در اینجا موجب افزایش حساسیت سیستم به ارتعاشات میکروسکوپی سطح پوست شده، که برای پایش بیماران در حالت استراحت کامل بسیار مناسب است.

\subsection{انتخاب پهنای باند} % Subsection 3.1.2
\label{sec:bandwidth-selection}

پهنای باند در رادار \lr{FMCW} به‌طور مستقیم با دقت تفکیک فاصله‌ای (\lr{$\Delta R$}) در ارتباط است. رابطه‌ی آن به‌صورت زیر تعریف می‌شود:
\[
\Delta R = \frac{c}{2B}
\]
\addtoequationlist{رابطه تفکیک فاصله} % Add this to your list of equations
\label{eq:range_resolution_fmcw} % Unique label for this specific equation instance

که در آن:
\begin{itemize}
    \item \textbf{\lr{$\Delta R$}}: دقت تفکیک فاصله،
    \item \textbf{\lr{$B$}}: پهنای باند،
    \item \textbf{\lr{$c$}}: سرعت نور است.
\end{itemize}

در مقاله "\lr{A Novel Scheme...}"، از پهنای باند وسیع (در حد گیگاهرتز) استفاده شده است تا بتوان سیگنال‌های ضربان قلب را از میان نویزها و حرکت‌های دیگر بدن تفکیک کرد. پهنای باند بالا همچنین باعث کاهش هم‌پوشانی سیگنال‌های نزدیک به هم (مانند حرکات قفسه سینه ناشی از تنفس) می‌شود.

در مقاله "\lr{Short-time Heart Rate Estimation...}" نیز به همین موضوع پرداخته شده، با این تفاوت که در شرایط محدود زمانی، نیاز به پهنای باندی بهینه برای تحلیل سریع در مدت‌زمان کوتاه نیز احساس می‌شود. لذا در این مقاله، علاوه بر پهنای باند مناسب، از تکنیک‌های فشرده‌سازی و فیلترینگ نیز برای بهبود دقت در تحلیل سیگنال بهره گرفته شده است.

در مطالعه "\lr{Continuous Radar-based...}"، با بهره‌گیری از رادار \lr{140 گیگاهرتز} و پهنای باند بالا، سیستم قادر به تفکیک دقیق تغییرات ضربان قلب حتی در شرایط بالینی بوده است. نتیجه این بود که افزایش پهنای باند به همراه استفاده از الگوریتم‌های خودهمبستگی فازی (\lr{Autocorrelation of Phase}) موجب شد تا دقت بسیار بالایی در محیط‌های واقعی حاصل شود.

\subsubsection*{جمع‌بندی} % Use subsubsection if you want numbering (3.1.3), or just a bold heading without numbering
\label{sec:summary-freq-bandwidth}

در طراحی سیستم رادار \lr{FMCW} برای اندازه‌گیری ضربان قلب، انتخاب فرکانس و پهنای باند به‌گونه‌ای باید انجام شود که تعادل مناسبی بین دقت، برد، حساسیت، و قابلیت عملیاتی در محیط‌های واقعی فراهم گردد. با توجه به مطالعات بررسی‌شده، استفاده از باندهای \lr{60–140 گیگاهرتز} به همراه پهنای باند وسیع و الگوریتم‌های پردازش تطبیقی می‌تواند دقت اندازه‌گیری را به سطح قابل قبول کاربردهای پزشکی برساند.


% ==============================================================================================================
\section{معماری سخت‌افزاری} % Section 3.2
\label{sec:hardware-architecture}

معماری سخت‌افزاری سیستم اندازه‌گیری ضربان قلب مبتنی بر رادار \lr{FMCW} نقش اساسی در دقت، پایداری و عملکرد سیستم دارد. دو بخش کلیدی در این معماری عبارتند از: ماژول رادار و آنتن‌ها و ماژول کنترل و ضبط سیگنال که هرکدام به‌نحوی در بهینه‌سازی برداشت داده‌های زیستی دخیل‌اند.

\subsection{ماژول رادار و آنتن} % Subsection 3.2.1
\label{sec:radar-module-antenna}

در طراحی سیستم رادار \lr{FMCW} برای تشخیص ضربان قلب، انتخاب و پیکربندی ماژول رادار و آرایش آنتن‌ها بسیار حائز اهمیت است. برای دستیابی به دقت بالا در تشخیص ضربان‌های ضعیف در سطح قفسه سینه، سیستم باید دارای حساسیت بالا و قابلیت تفکیک مکانی و جهتی دقیق باشد.
مطابق مقاله "\lr{Multi-Target Vital Signs Remote Monitoring Using mmWave FMCW Radar}"، استفاده از سیستم رادار \lr{MIMO} در باند \lr{60 گیگاهرتز} به همراه چند آنتن ارسال و دریافت، امکان تفکیک سیگنال‌های چند فرد در یک محیط مشترک را فراهم کرده است. در این سیستم، آنتن‌ها به‌صورت آرایه‌ای قرار گرفته‌اند تا از طریق تکنیک‌های پردازش فضایی، حرکت قفسه سینه هر فرد به‌صورت مجزا استخراج شود. این مقاله نشان می‌دهد که آرایش صحیح آنتن‌ها (\lr{TX} و \lr{RX}) باعث کاهش تداخل سیگنال‌ها و افزایش دقت جداسازی ضربان قلب افراد مختلف می‌شود.
در مقاله "\lr{A Robust and Accurate FMCW MIMO Radar Vital Sign Monitoring Framework...}"، از پیکربندی \lr{2} فرستنده و \lr{4} گیرنده (\lr{2T4R}) در باند \lr{K} استفاده شده است. همچنین، یک \lr{Beamformer} چهاربعدی (برد، زاویه، ارتفاع، داپلر) پیاده‌سازی شده که تمرکز سیگنال را روی ناحیه قلب فرد انجام می‌دهد. با این تکنیک، سیگنال ضربان قلب حتی در حضور تنفس و نویز محیطی با دقت بالا استخراج می‌شود. این نشان می‌دهد که در محیط‌هایی با \lr{SNR} پایین یا سیگنال ضعیف ضربان قلب، استفاده از آرایه‌های پیشرفته آنتن و \lr{beamforming} می‌تواند نقش کلیدی داشته باشد.

\subsubsection*{جمع‌بندی این بخش:} % Custom heading for summary
\begin{itemize}
    \item \textbf{استفاده از آرایه‌های چندآنتنه (\lr{MIMO}) به همراه الگوریتم‌های \lr{beamforming} پیشرفته، موجب تفکیک بهتر اهداف نزدیک و کاهش تداخل حرکات مختلف بدن می‌شود.}
    \item \textbf{فرکانس بالا (\lr{60–77 GHz}) و بهره‌گیری از آنتن‌های جهت‌دار در ساختار آرایه‌ای، دقت مکانی و حساسیت به ارتعاشات سطحی را افزایش می‌دهد.}
\end{itemize}
% ==============================================================================================================
\subsection{ماژول رادار و آنتن} % Subsection 3.2.1
\label{sec:radar-module-antenna}
% Add your text content for this subsection here.

% ==============================================================================================================
\subsection{مدول کنترل و ضبط سیگنال} % Subsection 3.2.2
\label{sec:control-signal-module}
% Add your text content for this subsection here.

% ==============================================================================================================
\section{نرم‌افزار پردازش اولیه} % Section 3.3
\label{sec:initial-processing-software}

پس از دریافت سیگنال‌های خام توسط رادار \lr{FMCW}، برای استخراج دقیق ضربان قلب، یک مرحله حیاتی شامل پردازش اولیه نرم‌افزاری انجام می‌شود. این مرحله شامل کالیبراسیون سیگنال، تقویت نسبت سیگنال به نویز (\lr{SNR})، حذف نویز پس‌زمینه، و فیلتر کردن مؤلفه‌های تنفسی یا حرکتی است.

\subsection{\lr{SNR}، کالیبراسیون و تنظیم} % Subsection 3.3.1
\label{sec:snr-calibration-adjustment}

در سیستم‌های رادار \lr{FMCW}، به دلیل وجود نویز محیطی، تداخل ناشی از حرکت‌های بدن و سیگنال‌های تنفسی، سیگنال ضربان قلب بسیار ضعیف است. بنابراین یکی از مراحل حیاتی در پردازش اولیه، افزایش \lr{SNR} و کالیبراسیون سیگنال ورودی است.

مقاله "\lr{A Novel Scheme of High-Precision Heart Rate Detection with a mm-Wave FMCW Radar}" پیشنهاد می‌کند که با استفاده از انتخاب تطبیقی \lr{bin} با بیشترین واریانس (\lr{adaptive bin selection}) و سپس استفاده از \lr{matched filtering} و \lr{Double-CZT frequency estimation} می‌توان سیگنال ضربان قلب را از بین داده‌های نویزی استخراج کرد. این روش باعث بهبود چشمگیر \lr{SNR} و کاهش اثر تنفس و حرکت شده است. در این مطالعه، \lr{MAE} کمتر از \lr{1 bpm} گزارش شده است که نشان‌دهنده دقت بالای این پردازش اولیه است.

همچنین، مقاله "\lr{HeartBeatNet: Enhancing Fast and Accurate Heart Rate Estimation With FMCW Radar and Lightweight Deep Learning}" از یک شبکه عصبی سبک به نام \lr{HeartBeatNet} برای شناسایی الگوهای ضربان قلب در \lr{spectrogram} رادار استفاده می‌کند. این شبکه نه‌تنها \lr{SNR} را به‌طور ضمنی از طریق استخراج ویژگی‌های قوی افزایش می‌دهد، بلکه قادر است به‌صورت بلادرنگ (\lr{real-time}) نرخ ضربان قلب را با دقت بالا تخمین بزند. این نشان می‌دهد که روش‌های مبتنی بر یادگیری ماشین نیز در تقویت سیگنال‌های ضعیف و افزایش \lr{SNR} مؤثرند.

**نکات کلیدی در بهبود \lr{SNR} و کالیبراسیون:**
\begin{itemize}
    \item \textbf{استفاده از \lr{matched filtering} و تخمینگر فرکانس دقیق (\lr{CZT})}
    \item \textbf{انتخاب آگاهانه دامنه‌های رنج راداری با بیشترین سیگنال مفید}
    \item \textbf{یادگیری الگوی ضربان با شبکه عصبی سبک (\lr{HeartBeatNet})}
    \item \textbf{حذف اثرات حرکت با مدل‌سازی دقیق طیف فرکانسی}
\end{itemize}

\subsection{حذف نویز پس‌زمینه} % Subsection 3.3.2
\label{sec:background-noise-removal}

پس از مرحله تقویت سیگنال، بخش مهم بعدی در پردازش اولیه، حذف نویز پس‌زمینه (\lr{clutter suppression}) است. این نویز می‌تواند شامل سیگنال‌های ثابت محیطی، حرکات بدن، یا تغییرات ناشی از تنفس باشد که ممکن است سیگنال ضربان قلب را مخفی کند.

در مقاله "\lr{Short-time Heart Rate Estimation Based on 60-GHz FMCW Radar}"، ابتدا میانگین‌گیری و فیلتر میانه روی داده‌ها اعمال شده تا نویز استاتیک حذف شود، سپس از تکنیک‌های \lr{ICEEMDAN} (نسخه تعمیم‌یافته تحلیل تجربی مدها) و \lr{Fast-ICA} برای تجزیه مؤلفه‌های تنفسی و استخراج مؤلفه ضربان قلب استفاده شده است. این رویکرد باعث شد تا ضربان قلب در پنجره‌های \lr{5 ثانیه‌ای} با دقت کمتر از \lr{4 bpm} قابل اندازه‌گیری باشد.

در مقاله "\lr{Improving Heart-Rate Estimation Accuracy Using Two-Beam Technique in Millimeter-Wave Vital-Sign Radar}"، نویسندگان با استفاده از تکنیک دو پرتوی هم‌پوشان (\lr{two-beam subtraction}) که مبتنی بر اختلاف فاز بین دو مسیر سیگنال است، موفق شدند تا اثر حرکات کلی بدن را حذف کرده و سیگنال ضربان را با دقت بالاتری استخراج کنند. این روش با مقایسه فاز تفاضلی دو کانال، نویزهای ناشی از حرکت را حذف می‌کند و سیگنال ضربان قلب را برجسته می‌سازد.

**نکات کلیدی در حذف نویز پس‌زمینه:**
\begin{itemize}
    \item \textbf{حذف نویز استاتیک با فیلترهای ساده (میانگین، میانه)}
    \item \textbf{جداسازی سیگنال ضربان از تنفس با \lr{ICEEMDAN} + \lr{Fast-ICA}}
    \item \textbf{حذف نویز حرکتی با تکنیک دو پرتوی (\lr{phase differential})}
    \item \textbf{استفاده از الگوریتم‌های \lr{decomposition} برای استخراج مؤلفه هدف}
\end{itemize}

% ==============================================================================================================
\subsection{کالیبراسیون و تنظیم \lr{SNR}} % Subsection 3.3.1
\label{sec:calibration-snr}
% Add your text content for this subsection here.

% ==============================================================================================================
\subsection{حذف نویز پس‌زمینه} % Subsection 3.3.2
\label{sec:background-noise-removal}
% Add your text content for this subsection here.

% ==============================================================================================================
\section{ملاحظات ایمنی و استانداردها} % Section 3.4
\label{sec:safety-standards}

استفاده از فناوری‌های مبتنی بر رادار \lr{FMCW} برای اندازه‌گیری غیرتماسی علائم حیاتی به‌ویژه ضربان قلب، نیازمند توجه دقیق به ملاحظات ایمنی تابش الکترومغناطیسی و رعایت استانداردهای بین‌المللی سلامت انسان است. از آنجا که در این سیستم‌ها از امواج میلی‌متری استفاده می‌شود، ارزیابی میزان تابش و اثرات بالقوه آن بر بدن انسان، بخش جدایی‌ناپذیر طراحی و پیاده‌سازی محسوب می‌شود.

\subsection{استانداردهای تابش الکترومغناطیسی} % Subsection 3.4.1
\label{sec:em-radiation-standards}

بنا بر دستورالعمل‌های کمیسیون بین‌المللی حفاظت در برابر تابش غیر یون‌ساز (\lr{ICNIRP})، برای فرکانس‌های بین \lr{2} تا \lr{300 گیگاهرتز}، حد مجاز چگالی توان تابشی برای تماس عمومی تقریباً \lr{10 وات بر متر مربع} است (\lr{ICNIRP, 2020}). این مقدار بر اساس مطالعات زیستی و اثرات گرمایی تابش بر بافت‌های انسانی تعیین شده و توسط نهادهایی مانند \lr{FCC} و اتحادیه اروپا نیز پذیرفته شده است.

در مقاله "\lr{A Novel FMCW Radar Scheme with Millimeter Motion Detection Capabilities...}" نویسندگان سیستم راداری خود را که با مدولاسیون \lr{SFMCW} در فرکانس \lr{24 گیگاهرتز} کار می‌کند، از نظر تابش الکترومغناطیسی ارزیابی کرده‌اند. آن‌ها با محاسبه چگالی توان در فواصل کاربردی (زیر \lr{1 متر}) نشان داده‌اند که توان تابشی رادار طراحی‌شده به‌مراتب کمتر از حد مجاز \lr{ICNIRP} است، و بنابراین برای استفاده طولانی‌مدت و بالینی کاملاً ایمن است. این تحلیل یکی از مهم‌ترین ارزیابی‌های عملی برای تایید کاربرد پزشکی رادار محسوب می‌شود.

\subsection{ایمنی در محیط بالینی و \lr{ICU}} % Subsection 3.4.2
\label{sec:clinical-icu-safety}

نصب و استفاده از سیستم‌های راداری در مراکز درمانی، به‌ویژه در بخش‌های حساس مانند \lr{ICU}، نیازمند اطمینان از عدم تداخل با تجهیزات پزشکی دیگر (مانند مانیتور قلب، پمپ تزریق یا سیستم‌های هشدار) است.

در مقاله "\lr{Continuous Radar-based Heart Rate Monitoring using Autocorrelation-based Algorithm in ICU}"، نویسندگان یک سیستم \lr{140 گیگاهرتزی FMCW} را زیر تخت بیماران قلبی نصب کرده‌اند. این تحقیق علاوه بر ارائه یک زنجیره پردازش دقیق برای استخراج ضربان قلب، به استفاده عملی و امن این رادار در محیط \lr{ICU} نیز پرداخته است. با گزارش خطای میانگین مطلق (\lr{MAE}) برابر با \lr{2.22 bpm} در \lr{15 بیمار}، و عدم بروز اختلال در عملکرد سایر تجهیزات پزشکی، نویسندگان به‌طور غیرمستقیم تأیید می‌کنند که طراحی سخت‌افزاری و تابش این رادار مطابق با الزامات ایمنی بیمارستانی بوده است. گرچه مقاله مستقیماً به استانداردهای تابشی اشاره نمی‌کند، عملکرد موفق در محیط بالینی نشان‌دهنده انطباق با شرایط ایمن محیط درمانی است.

\subsection{ملاحظات طراحی برای انطباق با ایمنی} % Subsection 3.4.3
\label{sec:design-considerations-safety}

برای طراحی سیستم راداری \lr{FMCW} قابل استفاده در حوزه سلامت، رعایت نکات زیر ضروری است:
\begin{itemize}
    \item \textbf{کاهش توان ارسال رادار به زیر حدود \lr{ICNIRP} با حفظ نسبت سیگنال به نویز (\lr{SNR}) مطلوب}
    \item \textbf{استفاده از پهنای باند محدود در محیط‌های بسته برای جلوگیری از تداخل بین‌دستگاهی}
    \item \textbf{ارزیابی عددی چگالی توان با روش‌های شبیه‌سازی \lr{EM} (مانند \lr{HFSS} یا \lr{CST})}
    \item \textbf{تست عدم تداخل با تجهیزات پزشکی (\lr{EMC testing})}
    \item \textbf{استفاده از آنتن‌هایی با الگوی تشعشعی محدود برای کنترل انتشار در محیط بسته}
\end{itemize}

\subsubsection*{جمع‌بندی} % Custom heading for summary
\label{sec:safety-summary}

با استناد به پژوهش‌های اخیر، مشخص شده است که رادارهای \lr{FMCW} در صورت طراحی صحیح، قابلیت استفاده ایمن در کاربردهای پزشکی از جمله \lr{ICU} را دارند. رعایت استانداردهای تابش، توان ارسال محدود، و استفاده از تکنیک‌های مدرن پردازش سیگنال، این امکان را فراهم می‌آورد تا این فناوری به‌عنوان جایگزینی غیرتماسی و کم‌ریسک در پایش علائم حیاتی مورد استفاده قرار گیرد.
