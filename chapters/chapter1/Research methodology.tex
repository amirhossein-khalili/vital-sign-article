\section{روش‌شناسی تحقیق} % Corresponds to 1.4
\label{sec:methodology}

\subsection{مقدمه} % Corresponds to 1.4.1
\label{sec:methodology-intro}
روش‌شناسی تحقیق در این مطالعه، شامل طراحی و پیاده‌سازی یک سیستم اندازه‌گیری ضربان قلب مبتنی بر رادار \lr{FMCW} است. این تحقیق به‌منظور ارزیابی دقت، عملکرد و قابلیت‌های رادارهای \lr{FMCW} برای اندازه‌گیری ضربان قلب در شرایط مختلف محیطی و عملیاتی طراحی شده است. در این فصل، به‌تفصیل روش‌ها و مراحل مختلف تحقیق شامل طراحی سیستم، نحوه جمع‌آوری داده‌ها، تحلیل داده‌ها و ارزیابی عملکرد سیستم پرداخته خواهد شد.

\subsection{طراحی سیستم راداری \lr{FMCW}} % Corresponds to 1.4.2
\label{sec:fmcw-system-design}
در ابتدا، یک سیستم راداری \lr{FMCW} برای اندازه‌گیری ضربان قلب طراحی می‌شود. این سیستم شامل اجزای زیر است:
\begin{enumerate}
    \item \textbf{منبع سیگنال راداری}: یک سیگنال مدولاسیون فرکانسی پیوسته تولید می‌شود که فرکانس آن به‌طور پیوسته تغییر می‌کند.
    \item \textbf{آنتن راداری}: برای ارسال و دریافت سیگنال‌های رادیویی به‌کار می‌رود.
    \item \textbf{مدار پردازش سیگنال}: سیگنال‌های دریافتی از آنتن پردازش شده و تغییرات ناشی از ضربان قلب استخراج می‌شود.
    \item \textbf{سیستم تحلیل داده‌ها}: پس از پردازش سیگنال، الگوریتم‌های مختلف برای تحلیل و استخراج ضربان قلب از داده‌های راداری استفاده می‌شود.
\end{enumerate}

\subsection{جمع‌آوری داده‌ها} % Corresponds to 1.4.3
\label{sec:data-collection}
برای جمع‌آوری داده‌ها، آزمایش‌های مختلف با استفاده از رادار \lr{FMCW} در محیط‌های متفاوت طراحی می‌شود. داده‌ها از دو بخش اصلی استخراج می‌شوند:
\begin{itemize}
    \item **داده‌های راداری**: این داده‌ها از سیگنال‌های بازتاب‌شده از بدن انسان که تحت تأثیر ضربان قلب تغییر می‌کنند، به‌دست می‌آید.
    \item **داده‌های مرجع**: برای مقایسه دقت سیستم راداری، از روش‌های سنتی اندازه‌گیری ضربان قلب مانند الکتروکاردیوگرام (\lr{ECG}) یا پالس اکسیمتر به‌عنوان داده‌های مرجع استفاده می‌شود.
\end{itemize}
آزمایش‌ها در شرایط مختلف انجام خواهد شد، از جمله در حالت ایستاده، نشسته و در حال حرکت، تا قابلیت اندازه‌گیری ضربان قلب در شرایط عملیاتی متفاوت ارزیابی گردد.

\subsection{(این بخش را باید بازنویس کنم )تحلیل داده‌ها و پردازش سیگنال} % Corresponds to 1.4.4 - Note: Your numbering was '4.' but should be '1.4.4' as a subsection.
\label{sec:data-analysis-signal-processing}
برای استخراج ضربان قلب از سیگنال‌های راداری، نیاز به استفاده از تکنیک‌های پردازش سیگنال است. این مرحله شامل چندین مرحله مهم است:
\begin{itemize}
    \item **فیلتر کردن سیگنال‌ها**: سیگنال‌های راداری ممکن است حاوی نویزهایی باشند که باید فیلتر شوند. از فیلترهای دیجیتال برای حذف نویزهای اضافی و بهبود دقت اندازه‌گیری استفاده می‌شود.
    \item **تبدیل فوریه**: سیگنال‌های راداری برای استخراج فرکانس تغییرات ضربان قلب به تبدیل فوریه نیاز دارند. این تبدیل به ما کمک می‌کند تا فرکانس ضربان قلب را از سیگنال‌های راداری استخراج کنیم.
    \item **تحلیل زمان-فرکانس**: برای شبیه‌سازی دقیق‌تر رفتار ضربان قلب، از روش‌های زمان-فرکانس مانند تبدیل ویولت برای استخراج ویژگی‌های دقیق‌تر استفاده می‌شود.
\end{itemize}

\subsubsection{ارزیابی دقت و مقایسه با روش‌های سنتی} % This corresponds to 1.4.4.1 (as it's a sub-part of 1.4.4)
\label{sec2}
در این مرحله، دقت سیستم راداری \lr{FMCW} با استفاده از داده‌های مرجع مقایسه می‌شود. برای ارزیابی دقت اندازه‌گیری ضربان قلب، معیارهای زیر در نظر گرفته می‌شوند:
\begin{itemize}
    \item **خطای مطلق اندازه‌گیری (\lr{MAE})**: تفاوت بین ضربان قلب اندازه‌گیری شده توسط رادار \lr{FMCW} و داده‌های مرجع.
    \item **دقت اندازه‌گیری**: درصد انطباق اندازه‌گیری‌ها با داده‌های مرجع.
    \item **پایداری اندازه‌گیری**: ارزیابی اینکه آیا سیستم در شرایط مختلف محیطی، از جمله حرکت بدن یا تغییرات در فاصله، قادر به حفظ دقت خود است یا خیر.
\end{itemize}

\subsection{تحلیل نتایج و بررسی شرایط مختلف} % Corresponds to 1.4.5
\label{sec:results-analysis}
در این بخش، نتایج به‌دست‌آمده از آزمایش‌ها تجزیه و تحلیل می‌شود. نتایج عملکرد رادار \lr{FMCW} در شرایط مختلف مانند ایستاده، نشسته و در حال حرکت، به‌طور جامع بررسی می‌شود. همچنین تأثیر عواملی مانند فاصله، موانع فیزیکی و شرایط محیطی (مانند حضور افراد دیگر یا نویز رادیویی) بر دقت اندازه‌گیری ضربان قلب بررسی خواهد شد.

\subsection{بهینه‌سازی سیستم} % Corresponds to 1.4.6
\label{sec:system-optimization}
در نهایت، بر اساس نتایج به‌دست‌آمده، بهینه‌سازی‌هایی برای سیستم پیشنهاد می‌شود. این بهینه‌سازی‌ها می‌تواند شامل بهبود الگوریتم‌های پردازش سیگنال، بهبود حساسیت راداری، و کاهش اثرات محیطی بر دقت اندازه‌گیری باشد.


