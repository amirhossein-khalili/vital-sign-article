\section{اهمیت و نوآوری تحقیق}\label{sec1}
\subsection{اهمیت تحقیق}\label{sec1}

با پیشرفت‌های روزافزون در تکنولوژی‌های ارتباطی و پردازشی، اندازه‌گیری ضربان قلب به عنوان یک پارامتر حیاتی در تشخیص وضعیت سلامت انسان، همواره در مرکز توجه محققان و پزشکان قرار داشته است. در این راستا، روش‌های سنتی اندازه‌گیری ضربان قلب که معمولاً به تجهیزات خاص یا تماس فیزیکی با بدن مانند الکتروکاردیوگرام (\lr{ECG}) یا پالس اکسیمتر وابسته هستند، می‌توانند برای بیماران دردناک باشند یا حتی در بعضی موارد عملی نباشند. به‌ویژه در محیط‌های بیمارستانی شلوغ یا در هنگام مراقبت‌های از راه دور، این روش‌ها ممکن است کارآیی و دقت کافی نداشته باشند.
با توجه به این محدودیت‌ها، استفاده از تکنولوژی‌های جدید مانند رادارهای \lr{FMCW} به عنوان یک راهکار دقیق برای اندازه‌گیری ضربان قلب، اهمیت ویژه‌ای پیدا کرده است. رادارهای \lr{FMCW} با استفاده از امواج رادیویی قادر به شبیه‌سازی حرکت‌های بسیار کوچک در بدن انسان، مانند تغییرات ناشی از ضربان قلب، هستند. این قابلیت‌ها باعث می‌شود که رادارهای \lr{FMCW} به ابزاری مؤثر در زمینه مراقبت‌های پزشکی از راه دور، پایش سلامت بیماران و حتی در محیط‌های پرتحرک تبدیل شوند.
این تحقیق می‌تواند با بررسی و ارزیابی کاربرد رادارهای \lr{FMCW} در اندازه‌گیری ضربان قلب، کمک شایانی به گسترش کاربرد این تکنولوژی در حوزه‌های پزشکی و بهداشت عمومی کند. با توجه به عدم نیاز به تماس فیزیکی و قابلیت اندازه‌گیری ضربان قلب در شرایط متنوع، این تحقیق می‌تواند زمینه‌ساز تحول در روش‌های نظارت بر سلامت انسان باشد.

\subsection{نوآوری تحقیق}\label{sec2}

نوآوری این تحقیق در استفاده از رادارهای \lr{FMCW}  برای اندازه‌گیری ضربان قلب به‌صورت غیرتماسی و در شرایط مختلف محیطی است. این نوآوری شامل جنبه‌های زیر می‌باشد:
\begin{enumerate}
    \item \textbf{استفاده از رادارهای \lr{FMCW} برای اندازه‌گیری ضربان قلب}: رادارهای \lr{FMCW} تا به امروز بیشتر در زمینه‌های نظامی، فضایی و صنعتی کاربرد داشته‌اند، اما این تحقیق اولین گام در استفاده از این رادارها برای اندازه‌گیری دقیق ضربان قلب انسان است. استفاده از این رادارها به‌عنوان ابزاری دقیق برای پایش سلامت، یک نوآوری در علوم پزشکی به‌شمار می‌رود.
    \item \textbf{اندازه‌گیری ضربان قلب در محیط‌های پرتحرک}: از آنجایی که رادارهای \lr{FMCW} قادر به اندازه‌گیری حرکت‌های بسیار کوچک در بدن هستند، این تحقیق به بررسی توانایی این رادارها در اندازه‌گیری ضربان قلب در شرایطی مانند حرکت بدن، فاصله، یا محیط‌های شلوغ و پرتحرک پرداخته است. این ویژگی می‌تواند تحولی در مراقبت‌های پزشکی از راه دور و نظارت بر وضعیت بیماران در هنگام فعالیت‌های روزمره باشد.
    \item \textbf{کاربرد رادارهای \lr{FMCW} در پایش سلامت از راه دور}: این تحقیق همچنین به بررسی کاربردهای رادارهای \lr{FMCW} در پایش ضربان قلب بیماران از راه دور پرداخته است. در این زمینه، رادارهای \lr{FMCW} می‌توانند به‌عنوان یک ابزار نظارتی بی‌دردسر و دقیق در مراکز درمانی و بیمارستان‌ها، بدون نیاز به تجهیزات اضافی یا تماس فیزیکی، عمل کنند.
    \item \textbf{توسعه الگوریتم‌های پردازش سیگنال برای بهبود دقت اندازه‌گیری}: یکی دیگر از جنبه‌های نوآورانه این تحقیق، طراحی و بهینه‌سازی الگوریتم‌های پردازش سیگنال برای تحلیل داده‌های رادار \lr{FMCW} به منظور افزایش دقت اندازه‌گیری ضربان قلب در شرایط مختلف محیطی است. این الگوریتم‌ها می‌توانند تغییرات کوچک در امواج راداری ناشی از ضربان قلب را از سایر سیگنال‌ها تفکیک کرده و دقت اندازه‌گیری را بهبود بخشند.
\end{enumerate}
