\section{مقدمه}\label{sec:intro-cap3}

% در سال‌های اخیر، نظارت مداوم بر علائم حیاتی انسان (ضربان قلب و نرخ تنفس) به‌عنوان ابزاری کلیدی در تشخیص به‌موقع حوادث حاد و پیش‌بینی زودهنگام وخامت شرایط بالینی مطرح شده است. روش‌های مرسوم مبتنی بر تماس، نظیر \lr{ECG} و \lr{PPG}، اگرچه دقت بالایی ارائه می‌دهند، اما به دلیل نیاز به تماس مستقیم با پوست، در شرایطی نظیر سوختگی، زخم‌های پوستی یا پاندمی‌ها می‌توانند کارآمدی و ایمنی کافی را نداشته باشند.

% در این راستا، رادارهای غیرتماسی مبتنی بر امواج \lr{میلی‌متری‌موج (mmWave)} به دلیل نفوذپذیری مطلوب در بافت‌های سطحی و مقاومت بالا در برابر تغییرات نوری و دمای محیط، به‌عنوان جایگزینی مناسب شناخته شده‌اند. میان انواع مختلف معماری‌های راداری، \lr{FMCW} به‌واسطه مدولاسیون فرکانس خطی، قابلیت تفکیک اهداف چندگانه و همزمان سنجش فاصله و حرکت را فراهم می‌آورد، که در پیاده‌سازی‌های غیرتماسی علائم حیاتی مزایای چشمگیری دارد.

% متون مروری اخیر نشان می‌دهند که تحقیقات متعددی به مقایسه و تحلیل معماری‌های \lr{CW}، \lr{FMCW} و \lr{UWB} پرداخته‌اند. برای نمونه، \lr{Liebetruth} و همکاران (2024) مروری سیستماتیک بر اندازه‌گیری ضربان قلب و تنفس با رادار ارائه داده‌اند و بر نقش متمایز مدولاسیون فرکانس در حذف اختلال‌های محیطی تأکید کرده‌اند. همچنین، \lr{Bin Obadi} و همکاران (2021) در یک مرور جدید، علاوه بر بررسی معماری‌های مختلف، به پیاده‌سازی نرم‌افزاری روی \lr{FPGA} و چالش‌های آن پرداخته‌اند.

% با این حال، اگرچه مطالعات بسیاری بر روی تخمین ضربان قلب و نرخ تنفس متمرکز شده‌اند، تحلیل جامعی از تأثیر پارامترهای فرکانس حامل و پهنای باند بر دقت اندازه‌گیری علائم حیاتی و نیز بررسی کامل جنبه‌های ایمنی و انطباق با استانداردهای پزشکی تا پیش از این کم‌تر انجام شده است. این شکاف پژوهشی انگیزه‌ای قوی برای ارائه یک بررسی متمرکز بر کاربرد رادارهای \lr{FMCW} در پزشکی ایجاد می‌کند.

% در ادامه، ابتدا مبانی فیزیکی و اصول کار رادارهای \lr{FMCW} همراه با بررسی تاریخچه کاربرد آن‌ها در پایش حیاتی بدن مرور خواهد شد. سپس تحولات کلیدی در الگوریتم‌های پردازش سیگنال و پروژه‌های عملی پیاده‌سازی سخت‌افزاری مورد بحث قرار می‌گیرند، تا زمینه‌ای روشن برای طراحی سیستم پیشنهادی این پایان‌نامه فراهم آید.


پایش مداوم علائم حیاتی به‌صورت غیرتماسی، چه در بخش‌های مراقبت ویژه و چه در پایش خانگی بیماران مزمن، نیازمند فناوری‌ای است که هم دقت بالای الکترودهای تماسی را داشته باشد و هم زحمت کمتری برای بیمار ایجاد کند. در میان سه معماری متداول راداری—\lr{CW}، \lr{UWB} و \lr{FMCW}—معماری \lr{FMCW} با ترکیب مدولاسیون خطی فرکانس و پردازش \lr{Beat Signal}، امکان اندازه‌گیری همزمان فاصله و ریزحرکت سطح قفسهٔ سینه را با پیچیدگی سخت‌افزاری قابل قبولی فراهم می‌کند؛ در نتیجه، خطاهای نقطهٔ کور \lr{CW} و محدودیت برد \lr{UWB} تا حد زیادی برطرف می‌شوند.

این فصل ابتدا نشان می‌دهد که چگونه پهنای باند مدولاسیون بر تفکیک‌پذیری فاصله اثر می‌گذارد:

\begin{equation}
\Delta R \approx \frac{c}{2B}
\label{eq:range_resolution}
\end{equation}
\addequation{اثر پهنای باند مدولاسیون بر تفکیک‌پذیری فاصله در سامانه‌های راداری}

و چرا بازه‌ی \lr{1–4 GHz} برای فواصل پزشکی \lr{0.5} تا \lr{3} متر کافی است. سپس به این نکته پرداخته می‌شود که افزایش فرکانس حامل در طیف میلی‌متری، طول موج را کوتاه و حساسیت فاز را افزایش می‌دهد؛ نتایج آزمایش روی رادارهای \lr{60 GHz} و \lr{120 GHz} نشان داده است که دقت اندازه‌گیری ضربان قلب در حدود $\pm 3.1$ \lr{bpm} و خطای تنفس کمتر از \lr{2 brpm} باقی می‌ماند، در حالی که سیستم \lr{24 GHz} به‌دلیل پروفایل نویز بالا کارایی ضعیف‌تری از خود نشان می‌دهد.

بر این مبنا، سخت‌افزار ارائه‌شده در این فصل بر پایه‌ی چیپ \lr{TI IWR1642} با آرایه‌ی مجتمع \lr{MIMO}، توان مصرفی کمتر از \lr{30 mW} و ساختار برد ارزیابی مدولار طراحی شده است؛ انتخابی که امکان جداسازی چند بیمار، اندازه‌ی کوچک، و سازگاری با باتری‌های پوشیدنی را فراهم می‌کند.

در ادامه، زنجیره‌ی پردازش شامل استخراج \lr{Beat Signal}، کالیبراسیون \lr{DC}، بازکردن فاز، فیلتر باندگذر تنفس و ضربان، سرکوب حرکات، و استفاده از الگوریتم \lr{Health-VMD} برای تخمین \lr{HRV} شرح داده می‌شود.

در نهایت، ملاحظات ایمنی، محدودیت توان تابشی، و مسیر اخذ مجوزهای بالینی تکمیل‌کننده‌ی چرخه‌ی طراحی سیستم به‌شمار می‌آیند.
