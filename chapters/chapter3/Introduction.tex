\section{مقدمه}\label{sec:intro-cap3}

در سال‌های اخیر، نظارت مداوم بر علائم حیاتی انسان (ضربان قلب و نرخ تنفس) به‌عنوان ابزاری کلیدی در تشخیص به‌موقع حوادث حاد و پیش‌بینی زودهنگام وخامت شرایط بالینی مطرح شده است. روش‌های مرسوم مبتنی بر تماس، نظیر \lr{ECG} و \lr{PPG}، اگرچه دقت بالایی ارائه می‌دهند، اما به دلیل نیاز به تماس مستقیم با پوست، در شرایطی نظیر سوختگی، زخم‌های پوستی یا پاندمی‌ها می‌توانند کارآمدی و ایمنی کافی را نداشته باشند.

در این راستا، رادارهای غیرتماسی مبتنی بر امواج \lr{میلی‌متری‌موج (mmWave)} به دلیل نفوذپذیری مطلوب در بافت‌های سطحی و مقاومت بالا در برابر تغییرات نوری و دمای محیط، به‌عنوان جایگزینی مناسب شناخته شده‌اند. میان انواع مختلف معماری‌های راداری، \lr{FMCW} به‌واسطه مدولاسیون فرکانس خطی، قابلیت تفکیک اهداف چندگانه و همزمان سنجش فاصله و حرکت را فراهم می‌آورد، که در پیاده‌سازی‌های غیرتماسی علائم حیاتی مزایای چشمگیری دارد.

متون مروری اخیر نشان می‌دهند که تحقیقات متعددی به مقایسه و تحلیل معماری‌های \lr{CW}، \lr{FMCW} و \lr{UWB} پرداخته‌اند. برای نمونه، \lr{Liebetruth} و همکاران (2024) مروری سیستماتیک بر اندازه‌گیری ضربان قلب و تنفس با رادار ارائه داده‌اند و بر نقش متمایز مدولاسیون فرکانس در حذف اختلال‌های محیطی تأکید کرده‌اند. همچنین، \lr{Bin Obadi} و همکاران (2021) در یک مرور جدید، علاوه بر بررسی معماری‌های مختلف، به پیاده‌سازی نرم‌افزاری روی \lr{FPGA} و چالش‌های آن پرداخته‌اند.

با این حال، اگرچه مطالعات بسیاری بر روی تخمین ضربان قلب و نرخ تنفس متمرکز شده‌اند، تحلیل جامعی از تأثیر پارامترهای فرکانس حامل و پهنای باند بر دقت اندازه‌گیری علائم حیاتی و نیز بررسی کامل جنبه‌های ایمنی و انطباق با استانداردهای پزشکی تا پیش از این کم‌تر انجام شده است. این شکاف پژوهشی انگیزه‌ای قوی برای ارائه یک بررسی متمرکز بر کاربرد رادارهای \lr{FMCW} در پزشکی ایجاد می‌کند.

در ادامه، ابتدا مبانی فیزیکی و اصول کار رادارهای \lr{FMCW} همراه با بررسی تاریخچه کاربرد آن‌ها در پایش حیاتی بدن مرور خواهد شد. سپس تحولات کلیدی در الگوریتم‌های پردازش سیگنال و پروژه‌های عملی پیاده‌سازی سخت‌افزاری مورد بحث قرار می‌گیرند، تا زمینه‌ای روشن برای طراحی سیستم پیشنهادی این پایان‌نامه فراهم آید.
