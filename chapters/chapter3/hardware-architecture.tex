\section{طراحی سخت‌افزار و معماری سیستم}

در این بخش ، جزئیات پیاده‌سازی سخت‌افزار سامانه مبتنی بر رادار \lr{FMCW} جهت اندازه‌گیری غیرتماسی علائم حیاتی تشریح می‌شود. ابتدا انتخاب چیپ‌ست راداری و پیکربندی آن بررسی می‌شود، سپس چگونگی ساختاربندی ماژول‌ها و ادغام آن‌ها در یک برد ارزیابی به همراه مشخصات آنتن و مصرف توان بیان می‌گردد. در نهایت، معماری \lr{MIMO} و مزایای آن در جداسازی چندشخص مورد بحث قرار می‌گیرد.

\subsection{انتخاب چیپ‌ست رادار و پیکربندی اولیه}

برای پیاده‌سازی رادار \lr{FMCW}، دو رویکرد اصلی مشاهده شده است:

\begin{itemize}
  \item چیپ‌های کم‌مصرف \lr{BGT} از شرکت \lr{Infineon} در فرکانس‌های ۲۴، ۶۰ و ۱۲۰ گیگاهرتز؛
  \item چیپ \lr{IWR1642} از شرکت \lr{Texas Instruments} در باند ۷۷ تا ۸۱  گیگاهرتز.
\end{itemize}
در مجموعه آزمایشات اولیه، سه رادار \lr{BGT24}، \lr{BGT60} و \lr{BGT120} با بهره‌گیری از بردهای ارزیابی ماژولار استفاده شدند. هر کدام از این سیستم‌ها مصرف توانی در بازه‌ی ۸ تا ۲۶ میلی‌وات دارند و توان خروجی آن‌ها در حدود \lr{5 dBm} تنظیم شده است. تمامی تجهیزات روی یک پنل آکریلیکی نصب شده و برای کنترل زاویه و فاصله به سه‌پایه مجهز شده‌اند.
\cite{marty2024frequency}

چیپ \lr{IWR1642} با باند فرکانسی \lr{77–81 GHz} و پهنای باند \lr{4 GHz}، به دلیل پشتیبانی داخلی از آرایه \lr{MIMO} و پردازش دیجیتال مجتمع، در طراحی سامانه نهایی این پایان‌نامه به‌عنوان پلتفرم اصلی انتخاب شده است. این چیپ ضمن فراهم آوردن امکان جداسازی زاویه‌ای، مصرف توان کمتر از \lr{30 mW} و نرخ نمونه‌برداری تا \lr{2 MHz} را پشتیبانی می‌کند.
\cite{song2023multi}

\subsection{برد ارزیابی و ساختار مدولار}
\label{sec:eval-board-structure}

برای مقایسه عملکرد بین سیستم‌ها و سهولت تنظیم پارامترها، هر رادار روی یک \textbf{برد ارزیابی ماژولار} نصب شد که دارای:

\begin{itemize}
  \item درگاه‌های پیکربندی دیجیتال (\lr{SPI/I\textsuperscript{2}C}) برای تنظیم فرکانس شروع ، پهنای باند ($B$) و نرخ چیپ؛
  \item مبدل \lr{ADC} با نرخ نمونه‌برداری \lr{2 MHz}؛
  \item خروجی \lr{USB/Serial} برای انتقال داده‌های \lr{IF} به رایانه جهت پردازش؛
  \item رابط آنتن خارجی یا بسته (\lr{Antenna-in-Package}) مطابق با فرکانس کاری؛
\end{itemize}

این بردها امکان تغییر سریع \textbf{پارامترهای چیپ} مانند شیب چیپ را فراهم می‌کنند:

\begin{equation}
S = \frac{B}{T_c}
\label{eq:chirp_slope}
\end{equation}
\addequation{رابطه شیب فرکانسی (\lr{Chirp Slope}) بر حسب پهنای باند و مدت زمان هر چیپ}

و همچنین تعداد چیپ‌ها در هر بازه زمانی (\lr{Chirp Density}) قابل تنظیم است.

\subsection{مشخصات آنتن و میدان دید}
\label{sec:antenna-fov}

رادارهای \lr{BGT60} و \lr{BGT120} از \textbf{آنتن داخل بسته} با بهره‌ی \lr{3.5 dBi} و نیم‌توان عرض پرتو $65^\circ$ در صفحه \lr{E} و $45^\circ$ در صفحه \lr{H} بهره می‌برند. رادار \lr{BGT24} دارای \textbf{آنتن خارجی} با مشخصات مشابه در فرکانس \lr{24 GHz} است. انتخاب پهنای باند \lr{2–10 GHz} برای \lr{BGT}های مختلف، به‌ترتیب موجب رزولوشن فاصله‌ای \lr{7.5}، \lr{3} و \lr{1.5} سانتی‌متر شد.

چیپ \lr{IWR1642} نیز از آنتن‌‌های مجتمع \lr{MIMO} (شامل ۳ فرستنده و ۴ گیرنده) برخوردار است که امکان تعیین زاویه ورود (\lr{Angle of Arrival}) و جداسازی چندشخصه را فراهم می‌سازد. با استفاده از ترکیب \lr{Beamforming} و روش‌های \lr{TDM-Orthogonalization}، آرایه مجازی ۱۲ کانالی قابل دستیابی است که وضوح زاویه‌ای به‌صورت زیر خواهد بود:

\begin{itemize}
  \item $\approx 16.6^\circ$ در افق (سمت-به-سمت)
  \item $\approx 45^\circ$ در ارتفاع (پایین-به-بالا)
\end{itemize}

\subsection{مدیریت مصرف توان}
\label{sec:power-management}
با توجه به کاربردهای پوشیدنی و همراه پزشکی، \textbf{مصرف توان} یکی از معیارهای حیاتی است. در پیکربندی پایه با نرخ چیپ \lr{100 Hz}، مصرف هر چیپ در حدود \lr{8 mW} برآورد شد که با فعال‌سازی کامل ماژول‌های پردازش دیجیتال، \lr{MIMO} و ارتباط بی‌سیم می‌تواند تا سقف \lr{26 mW} افزایش یابد. این مقدار همچنان در محدوده‌ی مجاز برای تجهیزات پوشیدنی پزشکی باقی می‌ماند و امکان تغذیه با باتری‌های سبک را فراهم می‌کند.
\cite{marty2024frequency}