\section{ملاحظات ایمنی و انطباق با استانداردهای پزشکی}
\label{sec:safety-standards}

در طراحی و پیاده‌سازی رادارهای \lr{FMCW} برای کاربردهای پزشکی غیرتماسی، رعایت استانداردهای ایمنی تابش الکترومغناطیسی و الزامات بالینی از اهمیت بالایی برخوردار است. این فصل به بررسی محدودیت‌های توان خروجی، معیارهای تشعشع و تداخل الکترومغناطیسی، و مستندات فنی مرتبط با تجهیزات پزشکی می‌پردازد.

\subsection{محدودیت توان خروجی و تابش سطحی}
\label{sec:power-density-limits}

طبق راهنمای \lr{ICNIRP} (کمیسیون بین‌المللی حفاظت در برابر تشعشع غیر‌یونیزه) و مقررات \lr{FCC} در ایالات متحده، میزان چگالی توان مجاز برای فرکانس‌های میلی‌متری‌موج در محدوده‌ی \lr{30–300 GHz} نباید از \lr{10 W/m²} در برخورد پوستی فراتر رود.

در سامانه‌های مبتنی بر چیپ‌های \lr{BGT} و \lr{IWR1642}، توان خروجی تنظیم‌شده حدود \lr{5 dBm} ($\sim$\lr{3.16 mW}) است که با توجه به آنتن‌های با بهره‌ی \lr{3–4 dBi} و فاصله‌ی عملیاتی \lr{0.5–3} متر، چگالی توان در سطح بدن فراتر از \lr{0.1 W/m²} نخواهد رفت. این مقدار به‌طور ایمن زیر حد \lr{ICNIRP} و \lr{FCC} قرار دارد و مطابق با توصیه‌های \lr{EPA} نیز ریسک زیستی محسوب نمی‌شود.

\subsection{معیارهای تشعشع الکترومغناطیسی و تداخل}
\label{sec:emc-sar}

برای حصول اطمینان از عدم تداخل با تجهیزات پزشکی حساس در بیمارستان‌ها (مثل مونیتورهای \lr{ECG} و پمپ‌های تزریق)، استانداردهای \lr{IEC 60601-1-2:2014} برای سازگاری الکترومغناطیسی (\lr{EMC}) باید رعایت شوند. این استاندارد مشخص می‌کند که دستگاه نباید میدان‌های ناخواسته با فرکانس‌های ناهمگن تولید کند و باید در برابر اختلالات محیطی تا سطح مشخص‌شده مقاوم باشد.

همچنین، استاندارد \lr{IEEE 1528:202X} (اصلاح‌شده برای رادارهای \lr{mmWave}) روش آزمون تابش را برای دستگاه‌های بردکوتاه پزشکی تعریف می‌کند. بر اساس آن، اندازه‌گیری \lr{SAR} (نرخ جذب ویژه) برای هر نقطه باید کمتر از \lr{2 W/kg} در بازه‌ی \lr{10} گرم بافت باشد. شبیه‌سازی‌های پراکندگی پرتو نشان داده‌اند که با توان خروجی \lr{5 dBm} و فاصله بیش از \lr{20} سانتی‌متر، \lr{SAR} به حدود \lr{0.05 W/kg} می‌رسد که به‌طور قابل‌توجهی زیر حد مجاز است.

\subsection{ایمنی بیماران و اپراتورها}
\label{sec:patient-operator-safety}

علاوه بر محیط بیمارستان، در کاربردهای بیمار در منزل نیز باید الزامات سادگی کاربری و هشدار خطر رعایت شود. طبق \lr{ISO 14971:2019} برای مدیریت خطر در تجهیزات پزشکی، باید ریسک‌های بالقوه ناشی از تابش الکترومغناطیسی، تداخل الکتریکی و ایمنی الکتریکی (\lr{IEC 60601-1:2005}) مستندسازی و تحلیل شوند.

\begin{itemize}
  \item \textbf{تابش امواج:} با توجه به توان پایین و ساختار بسته آنتن‌ها، خطر مستقیم تابش کاهش یافته است.
  \item \textbf{اختلال الکتریکی:} با قرار دادن فیلترهای \lr{EMI} و فریت‌بیرینگ روی خطوط تغذیه و سیگنال، پایداری الکترومغناطیسی حفظ می‌شود.
  \item \textbf{ایمنی برق:} رعایت استانداردهای حفاظت در برابر شوک الکتریکی (\lr{IEC 60601-1}) و استفاده از مدارهای جداساز (\lr{Isolator}) در بخش تغذیه \lr{USB} اطمینان می‌دهد که هیچ تماس مستقیم کاربر با سطوح ولتاژ بالا رخ ندهد.
\end{itemize}

\subsection{فرایند صدور گواهی و ثبت بالینی}
\label{sec:certification-process}

برای ورود به بازار تجهیزات پزشکی، دستگاه باید در چارچوب \lr{FDA} (ایالات متحده)، \lr{CE} (اروپا) و \lr{IR-MOH} (ایران) ثبت شود. مراحل اصلی عبارت‌اند از:

\begin{enumerate}
  \item \textbf{وثیقه‌نگاری فنی:} شامل گزارشات طراحی، آزمون \lr{EMC}، \lr{SAR} و تحلیل ریسک (\lr{ISO 14971}).
  \item \textbf{آزمون‌های کارآزمایی بالینی:} مطابق با \lr{ISO 14155}، باید مطالعه‌ای کنترل‌شده با حداقل ۲۰ بیمار انجام گیرد تا دقت و ایمنی دستگاه تأیید شود.
  \item \textbf{تهیه مستندات کیفی:} \lr{Quality Management System} مطابق \lr{ISO 13485} برای تضمین سازگاری تولید.
\end{enumerate}

این فرایندها معمولاً \lr{12–18} ماه زمان می‌برند اما برای سامانه‌های کم‌خطر (\lr{Class IIa} در اروپا) مسیر تسهیل‌شده (\lr{MDR}) وجود دارد که زمان ثبت را به حدود \lr{9} ماه کاهش می‌دهد.

\subsection*{جمع‌بندی بخش}
\label{sec:safety-summary}

با تنظیم توان خروجی پایین، طراحی مطابق با استانداردهای \lr{EMC} و \lr{SAR}، و مستندسازی فرایندهای مدیریت ریسک و ثبت بالینی، سامانه راداری موردنظر الزامات ایمنی و قانونی برای استفاده در محیط‌های بالینی و خانگی را تأمین می‌نماید.
