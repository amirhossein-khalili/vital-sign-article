
\section{مقدمه}
در این فصل به دو محور اصلی پرداخته می‌شود:

بررسی فیزیولوژی ضربان قلب و الگوی حرکتی با تأکید بر ویژگی‌های سیگنال‌های حرکتی (\lr{micro-motions}) قفسه سینه که ناشی از تپش‌های قلب هستند؛

مبانی نظری رادار \lr{FMCW (Frequency-Modulated Continuous-Wave)} و پارامترهای کلیدی طراحی آن تشریح شده و روش‌های پردازش پیشرفته برای جداسازی و اندازه‌گیری دقیق ضربان قلب به‌صورت غیرتماسی ارائه می‌شوند. در بخش اول، با مروری بر مکانیسم انقباض و انبساط میوکارد و انتشار موج پرفیوژن، نشان داده می‌شود که چگونه تغییرات بسیار کوچک موقعیت سطح قفسه سینه (در حد میلی‌متر یا کمتر) حامل اطلاعات حیاتی ضربان و نرخ متغیر قلب است. این تبیین فیزیولوژیک، مبنای انتخاب پارامترهای راداری مناسب را فراهم می‌آورد.

در بخش دوم، اصول کار رادارهای \lr{FMCW} با معرفی سیگنال‌های \lr{Chirp} و رابطه فرکانس \lr{Beat} برای اندازه‌گیری فاصله و سرعت توضیح داده خواهد شد
. سپس پارامترهای پهنای باند (\lr{B})، مدت زمان چرخه (\lr{Tp}) و نسبت سیگنال به نویز (\lr{SNR}) مورد بررسی قرار می‌گیرند که هر یک تأثیر مستقیمی بر دقت فاصله‌سنجی و تشخیص کوچک‌ترین حرکات دارند.

در ادامه، روش‌های متنوع پردازش سیگنال با هدف بهبود جداسازی ضربان قلب از نویز و حرکات مزاحم بدن مرور می‌شوند:
\begin{itemize}
  \item ارتقای تفکیک‌پذیری هم‌زمان فاصله و سرعت در رادارهای قابل پیکربندی
  \item مقایسه کارایی رادارهای داپلر و \lr{FMCW} در پایش علائم حیاتی غیرتماسی
  \item به‌کارگیری نمونه‌برداری انتخابی برای گسترش دامنه سرعت قابل تشخیص
  \item استفاده از رادار میلی‌متری جهت پایش هم‌زمان چند سوژه و کاهش نویز با کمک الگوریتم تجزیه مقادیر منفرد (\lr{SVD})
  \item تحلیل تغییرات درون‌فردی سیگنال‌های سیزموکاردیوگرام
  \item ارائه چارچوب‌های چندهدفه برای استخراج و پایش تغییرات ضربان قلب
\end{itemize}

این ساختار فصل، خواننده را از مبانی فیزیولوژیک تا روش‌های نوین 
پردازش سیگنال راداری هدایت می‌کند و زمینه را برای ارائه روش تحقیق و نتایج تجربی در فصل‌های بعدی فراهم می‌آورد.

