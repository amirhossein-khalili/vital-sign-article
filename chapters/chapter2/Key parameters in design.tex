\section{پارامترهای کلیدی در طراحی \lr{FMCW} (\lr{B}, \lr{$T_p$}, \lr{SNR} و ...)}\label{sec:fmcw-key-parameters} % Corresponds to 2.2


در سیستم‌های رادار \lr{FMCW}، طراحی بهینه و انتخاب مناسب پارامترهای مختلف تأثیر بسزایی در دقت اندازه‌گیری، عملکرد، و قابلیت تشخیص اهداف دارد. در این بخش، به بررسی پارامترهای کلیدی در طراحی سیستم‌های \lr{FMCW} پرداخته خواهد شد که به‌ویژه در کاربردهای پزشکی مانند اندازه‌گیری ضربان قلب اهمیت زیادی دارند. این پارامترها شامل پهنای باند (\lr{B})، مدت زمان چرخه (\lr{$T_p$})، نسبت سیگنال به نویز (\lr{SNR}) و برخی دیگر از فاکتورهای مهم در طراحی سیستم راداری هستند.

\subsection{پهنای باند (\lr{B})} % Corresponds to 2.2.1
\label{sec:bandwidth}

پهنای باند \lr{$B$} یکی از مهم‌ترین پارامترها در طراحی سیستم‌های راداری \lr{FMCW} است. افزایش پهنای باند منجر به دقت بالاتر در اندازه‌گیری فاصله و تفکیک بهتر اهداف می‌شود. طبق مقاله \cite{islam2020non}، با افزایش پهنای باند، سیستم قادر به تمایز بهتر میان اهداف نزدیک به هم خواهد بود که این ویژگی در اندازه‌گیری‌های پزشکی بسیار حائز اهمیت است، به‌ویژه برای تمایز ضربان قلب از دیگر سیگنال‌های جسمی.
رابطه‌ی پهنای باند با تفکیک فاصله به‌صورت زیر بیان می‌شود:
\begin{equation}
\Delta R = \frac{c}{2B}
\label{eq:range_resolution_bandwidth}
\end{equation}
\addequation{رابطه‌ی میان پهنای باند و تفکیک فاصله‌ای}

که در آن \lr{$\Delta R$} تفکیک فاصله و \lr{$c$} سرعت نور است. این رابطه نشان می‌دهد که با افزایش \lr{$B$}، تفکیک فاصله بهبود می‌یابد و دقت اندازه‌گیری افزایش می‌یابد. در اندازه‌گیری ضربان قلب، این ویژگی باعث می‌شود که سیستم بتواند تغییرات جزئی در حرکت بدن که ناشی از ضربان قلب است، با دقت بیشتری شبیه‌سازی کند.

\subsection{مدت زمان چرخه (\lr{$T_p$})} % Corresponds to 2.2.2
\label{sec:chirp-duration}

مدت زمان چرخه \lr{$T_p$} که به آن مدت زمان مدولاسیون نیز گفته می‌شود، تأثیر زیادی بر عملکرد رادارهای \lr{FMCW} دارد. مدت زمان مدولاسیون، مدت زمانی است که سیگنال فرکانس خطی تغییر می‌کند. طبق مقاله \cite{lv2024millimeter}، افزایش مدت زمان \lr{$T_p$} موجب افزایش دقت در شبیه‌سازی تغییرات ضربان قلب می‌شود، زیرا طولانی‌تر شدن چرخه سیگنال امکان تحلیل دقیق‌تر تغییرات سیگنال‌های برگشتی را فراهم می‌آورد.
با توجه به این مقاله، افزایش مدت زمان چرخه می‌تواند باعث بهبود توانایی سیستم در تمایز نویز از سیگنال اصلی ضربان قلب شود. همچنین، انتخاب مدت زمان مناسب برای مدولاسیون سیگنال، ارتباط مستقیمی با دقت اندازه‌گیری سرعت و تفکیک داپلر دارد.

\subsection{نسبت سیگنال به نویز (\lr{SNR})} % Corresponds to 2.2.3
\label{sec:snr}

یکی از چالش‌های اصلی در سیستم‌های راداری \lr{FMCW}، حفظ نسبت سیگنال به نویز (\lr{SNR}) مناسب است. \lr{SNR} یکی از عوامل مهمی است که کیفیت داده‌های دریافتی و دقت اندازه‌گیری را تحت تأثیر قرار می‌دهد. سیستم‌های راداری با \lr{SNR} پایین قادر به تشخیص دقیق تغییرات کوچک در سیگنال‌های برگشتی نخواهند بود، که این مشکل در اندازه‌گیری ضربان قلب در شرایط مختلف محیطی، به‌ویژه در محیط‌های پر نویز مانند بیمارستان‌ها یا محیط‌های شلوغ، به‌وضوح مشاهده می‌شود.
مقاله \cite{islam2020non} به این موضوع پرداخته و اشاره می‌کند که برای پایش علائم حیاتی به‌ویژه ضربان قلب، لازم است که \lr{SNR} به‌گونه‌ای تنظیم شود که سیگنال‌های مرتبط با ضربان قلب به وضوح از نویزهای دیگر تمایز یابند. یکی از روش‌های پیشنهادی برای بهبود \lr{SNR}، استفاده از تکنیک‌های پردازش سیگنال پیشرفته و فیلترینگ هوشمند است.
بهبود \lr{SNR} معمولاً با استفاده از افزایش پهنای باند (\lr{B})، بهبود الگوریتم‌های پردازش سیگنال و افزایش قدرت سیگنال‌های ارسالی امکان‌پذیر است. این کار می‌تواند باعث کاهش اثرات نویز محیطی بر روی داده‌های دریافتی شود.

\subsection{سایر پارامترهای طراحی} % Corresponds to 2.2.4
\label{sec:other-design-parameters}

علاوه بر پارامترهای فوق، در طراحی سیستم‌های \lr{FMCW} برای اندازه‌گیری ضربان قلب، دیگر پارامترها نیز باید در نظر گرفته شوند. برای مثال:
\begin{itemize}
    \item \textbf{آنتن راداری}: طراحی آنتن با قابلیت پوشش‌دهی مناسب برای اندازه‌گیری دقیق اهداف با فاصله‌های مختلف، یکی از ارکان کلیدی طراحی سیستم رادار \lr{FMCW} است.
    \item \textbf{توان سیگنال ارسالی}: میزان توان سیگنال ارسال‌شده می‌تواند بر کیفیت داده‌های دریافتی و حساسیت سیستم تأثیرگذار باشد.
    \item \textbf{پردازش سیگنال تطبیقی}: استفاده از الگوریتم‌های تطبیقی برای فیلتر کردن نویز و شبیه‌سازی دقیق‌تر ضربان قلب می‌تواند به افزایش دقت سیستم کمک کند.
\end{itemize}


